\chapter{现代控制论基础}
\section{状态空间方程与线性系统}
\begin{equation*}
	\begin{cases}
		x[k+1] = Ax[k] + Bu[k]\\
		y[k] = Cx[k] + Du[k]
	\end{cases}
\end{equation*}
\subsection{连续系统到离散系统相互转换}
线性情况下
\begin{equation*}
  \begin{cases}
  	\dot{x} = A_cx + B_cu\\
  	y = C_cx + D_cu
  \end{cases}
\end{equation*}
可以转换为
\begin{equation*}
  \begin{cases}
  	x[k+1] = A_dx[k] + B_du[k]\\
  	y[k] = C_dx[k] + D_du[k]
  \end{cases}
\end{equation*}
其中
\begin{gather*}
 A_d = I + A_c\Delta t\\
 B_d = B_c \Delta t \\
 C_d = C_c\\
 D_d = D_c 
\end{gather*}
\subsubsection{前向欧拉法}
\subsubsection{后向欧拉法}
\subsubsection{其他方法}
\section{非线性近似为线性系统}
\section{最小二乘推导}
测量公式,$\theta \in R^n$ 是隐藏参数,$ v \in R^m $ 是测量噪声
\begin{equation}
  y = g(\theta) + v
\end{equation}

问题定义:
\begin{equation}
  \min{\theta \in \Theta}J(\theta) = min \Vert y - g(\theta) \Vert ^2
\end{equation}
线性最小二乘:$g(\theta) = H \theta $
利用简单的数学计算
\begin{theorembox}
	if $f(x) = Ax$, then $\nabla f(x) = A$
\end{theorembox}
\begin{theorembox}
	if $f(x) = x^TAx$, then $\nabla f(x) = A^Tx+Ax$
\end{theorembox}
可以证明
\begin{theorembox}
	如果$H\in R^{m\times n}$列满秩,则存在解$\theta = (H^TH)^{-1}H^Ty$ 
\end{theorembox}
\section{逆矩阵}
\begin{theorembox}
	针对矩阵$A \in R^{m\times n}$,若A列满秩,则矩阵左逆为$A_{left}^{-1} = (A^TA)^{-1}A^T$,满足$A_{left}^{-1}A = I_{n \times n}$。
\end{theorembox}


\section{伪逆矩阵Moore-Penrose逆}
摩尔-彭罗斯逆的应用场景:
\begin{itemize}
  \item 最小二乘问题::线性回归等问题中,当系数矩阵不是方阵时,摩尔-彭罗斯逆可以用来求解最小二乘解。
  \item 欠定方程组::对于方程组的解不唯一的情况,摩尔-彭罗斯逆可以找到一个具有最小范数的解。
  \item 矩阵的秩的计算::摩尔-彭罗斯逆可以用来确定一个矩阵的秩。
  \item 信号处理和图像处理::在这些领域中,摩尔-彭罗斯逆可以用于去噪、滤波等任务。
  \item 控制理论::用于设计控制器,特别是当系统模型是非方阵时。
\end{itemize}
摩尔-彭罗斯逆的性质:
\begin{itemize}
  \item 唯一性::对于任意一个矩阵,其摩尔-彭罗斯逆是唯一的。
  \item 广义逆::如果一个矩阵是可逆的,那么它的摩尔-彭罗斯逆就是它的普通逆矩阵。
  \item 满足Penrose 方程::摩尔-彭罗斯逆满足四个特定的Penrose 方程,这些方程定义了摩尔-彭罗斯逆的性质。
\end{itemize}



\section{投影矩阵}

\subsection{最小二乘应用与例子}
\begin{enumerate}
  \item 可以用作求解线性方程组的最小
二乘根,使得解误差的而范数最小
  \item 可以通过构造方程组来进行系统参数辨识
\end{enumerate}
\section{稳定性、可控性和可观性}
\section{状态反馈/输出反馈控制设计}
\section{状态空间跟踪控制器设计}
\section{卡尔曼滤波原理}
\subsection{扩展卡尔曼滤波}




\chapter{函数最优化问题求解方式}
\section{原始牛顿法}
\section{雅可比矩阵和Hessian矩阵}
\section{高斯-牛顿法}
\section{拟牛顿法}
\subsection{Levenberg-Marquardt algorithm, LM}
\subsection{Davidon-Fletcher-Powell algoritm DFP}
\subsection{Broyden–Fletcher–Goldfarb–Shanno algorithm BFGS}


\chapter{最优控制}
\section{LQR}
\subsection{iLQR}
\subsection{DDP}

\chapter{轨迹优化}
\section{概述}
\section{直接法}
\subsection{Single Shooting}
\subsection{Multiple Shooting}
\subsection{Collocation Method}
\section{间接法}





