\chapter{现代控制论基础}
\section{状态空间方程与线性系统}
\begin{equation*}
	\begin{cases}
		x[k+1] = Ax[k] + Bu[k]\\
		y[k] = Cx[k] + Du[k]
	\end{cases}
\end{equation*}
\subsection{连续系统到离散系统相互转换}
线性情况下
\begin{equation*}
  \begin{cases}
  	\dot{x} = A_cx + B_cu\\
  	y = C_cx + D_cu
  \end{cases}
\end{equation*}
可以转换为
\begin{equation*}
  \begin{cases}
  	x[k+1] = A_dx[k] + B_du[k]\\
  	y[k] = C_dx[k] + D_du[k]
  \end{cases}
\end{equation*}
其中
\begin{gather*}
 A_d = I + A_c\Delta t\\
 B_d = B_c \Delta t \\
 C_d = C_c\\
 D_d = D_c 
\end{gather*}
\subsubsection{前向欧拉法}
\subsubsection{后向欧拉法}
\subsubsection{其他方法}
\section{非线性近似为线性系统}
\section{最小二乘}
测量公式,$\theta \in R^n$ 是隐藏参数,$ v \in R^m $ 是测量噪声
\begin{equation}
  y = g(\theta) + v
\end{equation}

问题定义:
\begin{equation}
  \min_{\theta \in \Theta}J(\theta) = min \Vert y - g(\theta) \Vert ^2
\end{equation}

\subsection{线性最小二乘}
有$g(\theta) = H \theta $
利用简单的数学计算
\begin{theorembox}
	如果有$f(x) = Ax$, 则有$\nabla f(x) = A$
\end{theorembox}
\begin{theorembox}
	如果有$f(x) = x^TAx$, 则有$\nabla f(x) = A^Tx+Ax$
\end{theorembox}
可以证明
\begin{theorembox}
	如果$H\in R^{m\times n}$列满秩,则存在解$\theta = (H^TH)^{-1}H^Ty$ 
\end{theorembox}
\subsection{带正则化的最小二乘}\label{ModernControl::sec::regularized-least-squares}
为了防止实际过程中的过拟合问题,常常把目标函数加上正则化项,此时目标函数变成
% todo: 有其他用途?比如行满秩时最小二乘
\begin{equation*}
	\min_{\theta \in \Theta} \Vert y - H\theta \Vert ^2 + \mu \Vert \theta \Vert ^2
\end{equation*}
\begin{theorembox}
	如上目标函数的最小二乘解为$\theta = (H^TH + \mu I)^{-1}H^Ty$。其中$\mu$为可调参数。且当$\mu \rightarrow 0$,有 $(H^TH + \mu I)^{-1}H^T \rightarrow H^T(HH^T)^{-1}$
\end{theorembox}
% todo: 添加推导过程
这也解释了矩阵右逆和带正则化的最小二乘之间的关系。
\subsection{上述方法(左伪逆法)求解最小二乘的缺陷}
用这个方法解决 least squares problems 存在两个缺陷:

\begin{itemize}
  \item $A^T A$ 会导致信息的丢失。假设 A 中某一项是一个计算机刚好能表示的浮点数,A 乘以 A 的转置后浮点数的平方可能超出精度而被丢失,从而导致 $A^T A$ 是奇异矩阵无法求解,应对的方法是后面要学的奇异值分解。
  \item $A^T A$ 的条件数是 A 的平方。系统不稳定性变大了。应对的方法是对数据进行中心化预处理,这样做的目的是要增加基向量的正交性。
\end{itemize}

\subsection{QR分解求最小二乘}
\subsection{QR分解用于更新最小二乘解集}
% todo : QR分解用于最小二乘

\begin{proofbox}[QR分解求解左逆矩阵(最小二乘)]
	
\end{proofbox}
% todo : SVD分解用于求解最小二乘
\subsection{SVD分解用于最小二乘求解}
\begin{theorembox}
	$\theta = V\Sigma ^{-1}U_1b$
\end{theorembox}

\section{逆矩阵}
\begin{theorembox}
	系统辨识常使用最小二乘法解决。线性最小二乘本质上是求解线性方程组。求解线性方程组本质是在求矩阵的逆。
	\begin{itemize}
  \item 当矩阵为高矩阵,列满秩,超定,方程可能无解,使用左逆可以求得最小二乘解。
  \item 另外一种情况是,矩阵为胖矩阵,行满秩,欠定,矩阵有无数多解,使用右逆可以求得最小范数解。
  \item Moore-Penrose提出了广义逆矩阵,在高满秩矩阵时是左逆,在胖满秩矩阵时是右逆。可以同刚果cholesky分解,QR分解,SVD分解等方式计算出广义逆。
  \item 
\end{itemize}
\end{theorembox}

\begin{lemmabox}[一些有意思的思考]
	\begin{itemize}
  \item 方程组本质上是自变量进行了空间变换得到了因变量。求解方程组任务常常是已知因变量求自变量。
  \item 线性化可以求解因变量在空间变换中的梯度。在一定尺度上可以将非线性系统转化为线性系统。
  \item 梯度下降法等迭代方法可以利用因变量梯度使自变量逐步逼近真实值。(系统需满足特定条件)
  \item 梯度下降法在单次迭代的过程中退化为求解线性方程组问题
\end{itemize}
\end{lemmabox}

\subsection{左逆矩阵求解最小二乘问题}
\begin{theorembox}
	针对矩阵$A \in R^{m\times n}$,若A列满秩,则矩阵左逆为$A_{left}^{-1} = (A^TA)^{-1}A^T$,满足$A_{left}^{-1}A = I_{n \times n}$。此时$(A^TA)^{-1}A^Ty$为最小二乘解
\end{theorembox}
\subsection{右逆矩阵求解欠定方程}
\begin{theorembox}
	针对矩阵$A \in R^{m\times n}$,若A行满秩,则矩阵右逆为$A_{right}^{-1} = A^T(AA^T)^{-1}$。此时$A^T(AA^T)^{-1}y$为无穷多解中的最小范数解。
\end{theorembox}
\subsubsection{右逆矩阵可以通过QR分解得到}

\begin{theorembox}[通过QR分解求解右逆矩阵]
	将矩阵A的转置进行QR分解,$A^T = QR$。有$Q\in R^{n\times m}$,$R\in R^{m\times m}$,R为上三角矩阵,非奇异。
	\begin{itemize}
  		\item $x_{ln} = A^T(AA^T)^{-1}y = QR^{-T}y$
  		\item $\Vert x_{ln}\Vert = \Vert R^{-T}y\Vert$
	\end{itemize}
\end{theorembox}

QR分解的方式参考\ref{MathTools:sec:qr_decomposition}节中的QR分解方法
\externaldocument{../MathTools/MathTools}
\subsubsection{QR分解可以通过HouseHolder法得到}
参考\ref{MathTools:sec:householder}节中的详细推导。
\subsection{伪逆矩阵Moore-Penrose逆}
% 引用MathTools中的矩阵逆相关内容
关于矩阵逆的详细讨论,参考\ref{MathTools:chap:matrix_inverse}章。

\subsubsection{摩尔-彭罗斯逆的应用场景}
\begin{itemize}
  \item 最小二乘问题::线性回归等问题中,当系数矩阵不是方阵时,摩尔-彭罗斯逆可以用来求解最小二乘解。
  \item 欠定方程组::对于方程组的解不唯一的情况,摩尔-彭罗斯逆可以找到一个具有最小范数的解。
  \item 矩阵的秩的计算::摩尔-彭罗斯逆可以用来确定一个矩阵的秩。
  \item 信号处理和图像处理::在这些领域中,摩尔-彭罗斯逆可以用于去噪、滤波等任务。
  \item 控制理论::用于设计控制器,特别是当系统模型是非方阵时。
\end{itemize}
\subsubsection{摩尔-彭罗斯逆的性质}
\begin{itemize}
  \item 唯一性::对于任意一个矩阵,其摩尔-彭罗斯逆是唯一的。
  \item 广义逆::如果一个矩阵是可逆的,那么它的摩尔-彭罗斯逆就是它的普通逆矩阵。
  \item 满足Penrose 方程::摩尔-彭罗斯逆满足四个特定的Penrose 方程,这些方程定义了摩尔-彭罗斯逆的性质。
\end{itemize}



\subsection{投影矩阵}
% todo: 添加投影矩阵的用法
投影矩阵的计算可以基于\ref{MathTools:chap:matrix_decomposition}章中的矩阵分解方法。

\subsection{最小二乘应用与例子}
\begin{enumerate}
  \item 可以用作求解线性方程组的最小
二乘根,使得解误差的而范数最小
  \item 可以通过构造方程组来进行系统参数辨识
\end{enumerate}
\section{状态空间解}
\begin{theorembox}
	\begin{equation*}
	  \begin{cases}
		x(k) = A^k\hat{x} + \Sigma^{k-1}_{j=0} A^{k-j-1}Bu(j) \\
		y(k) = CA^k\hat{x} + \Sigma^{k-1}_{j=0} CA^{k-j-1} Bu(j) + Du(k)
	  \end{cases}
    \end{equation*}
\end{theorembox}

一大部分的控制问题可以被转换成为Regulation Problem:让输入或者输出靠近某一个确定常数。

\section{渐进稳定性分析}

渐进稳定:内部零输入稳定
\begin{theorembox}
	LTI(A,B) 是渐进稳定的如果对于所有的特征值$|\lambda _i| < 1$
\end{theorembox}

BIBO:有界输入有界输出
\begin{theorembox}
	如果系统渐进稳定,那么系统BIBO稳定。若系统BIBO稳定,不一定渐进稳定。
\end{theorembox}
\section{可控性}
\begin{theorembox}
	Cayley Hamilton理论:对于任何一个矩阵$A \in R^{n\times n}$, $A^k$能写成{$I,A,A^2, ... , A^{n-1}$}的线性组合
\end{theorembox}
\begin{theorembox}
	可控性定义:系统能够在有限步的时间里,从任意一个初始状态到达任意一个状态。\\
	若矩阵$[B, AB, ..., A^{k-1}B]$满秩,则说明系统可控
\end{theorembox}
\section{可观性}
\begin{theorembox}
	可观性定义:若给定有限步的全控制序列和全状态序列,能够确定唯一的系统状态。\\
	若矩阵
	\begin{equation*}
		\begin{bmatrix}
		C &\\
		CA &\\
		\vdots &\\
		CA^{n-1} &
		\end{bmatrix}
	\end{equation*}
	满秩,则系统可观
\end{theorembox}
\section{相似系统下的系统特性不变}
\begin{theorembox}
	定义相似系统$z = P^{-1}x$, 有如下相似系统:
	\begin{equation*}
  \begin{cases}
  	\hat{A} = P^{-1}AP \\
  	\hat{B} = P^{-1}B \\
  	\hat{C} = CP \\
  	\hat{D} = D
  \end{cases}
\end{equation*}
相似系统的可控性和可观性不变
\end{theorembox}
\section{状态反馈/输出反馈控制设计}
\begin{theorembox}
	定义控制规范型如,若系统形如:
	\begin{equation*}
	\bar{A} = 
  \begin{bmatrix}
 0 & 1& \cdots & 0 \\
 \vdots & \vdots & \ddots & \\
 0 & \cdots & \cdots & 1 \\
 -\alpha_0 &-\alpha_1 & \cdots & -\alpha_{n-1}
\end{bmatrix}, \bar{B} = \begin{bmatrix}
	0 \\ \vdots \\ 0 \\ 1
\end{bmatrix}
\end{equation*}
则系统总是可控的。\\
其中$\bar{A}$的特征多项式为$\Delta_{\bar{A}}(\lambda)=\lambda ^n + \alpha_{n-1}\lambda^{n-1} + \cdots + \alpha_1 \lambda + \alpha_0$。且相似系统的特征多项式不变\\
% todo : 配置根法的反馈控制器设计
% todo : 闭环特征值选取策略
\end{theorembox}
\section{Luenberger状态观测器}
当系统状态不可知,系统输入只能依赖系统输出
\begin{theorembox}
	迭代的生成估计系统状态:
	\begin{equation*}
  \hat{x}(k+1) = A \hat{x}(k) + Bu(k) + L[y(k) - C\hat{x}(k) - Du(k)]
\end{equation*}
有系统稳定性分析:
\begin{equation*}
  e(k+1) = (A-LC)e(k)
\end{equation*}
控制器设计:对于可控可观系统,首先选择合适的反馈矩阵$(A-BK)$,以及比反馈矩阵收敛速度快很多的观测矩阵$(A-LC)$。使得$u(k+1) = -K\hat{x}(k+1)$
\end{theorembox}
\section{卡尔曼滤波原理}
卡尔曼滤波是基于概率论,从系统输出和输入得到系统状态的最优估计。
\begin{equation*}
  \begin{Bmatrix}
  	\hat{x}_k \\ P_k
  \end{Bmatrix}
  \longrightarrow
  \begin{Bmatrix}
  	\hat{x}_{k+1|k}\\ P_{k+1|k}
  \end{Bmatrix}
  \longrightarrow
  \begin{Bmatrix}
  	\hat{x}_{k+1} \\ P_{k+1}
  \end{Bmatrix}
\end{equation*}
\begin{theorembox}
\begin{itemize}
  \item 预测
  \begin{enumerate}
  \item 提前预测下一时刻状态
  	\begin{equation*}
  	\hat{x}_{k|k-1} = A_{k-1}\hat{x}_{k-1} + B_{k-1}u_{k-1}
	\end{equation*}
  \item 提前预测误差方差
  	\begin{equation*}
	P_{k|k-1}=A_{k-1}P_{k-1}A_{k-1}^T + Q_{k-1}  
	\end{equation*}
  \end{enumerate}
  \item 测量更新
  \begin{enumerate}
  	\item 计算卡尔曼增益
  	\begin{equation*}
  K_k = P_{k|k-1}C_k^T(C_kP_{k|k-1}C_k^T + R_k)^{-1}
\end{equation*}
  	\item 更新估计
  	\begin{equation*}
  		\hat{x}_{k} = \hat{x}_{k|k-1} + K_k(y_k - C_kx_{k|k-1} - D_ku_k)
  	\end{equation*}
  	\item 更新误差方差
  	\begin{equation*}
  		P_k = (I - K_kC_k)P_{k|k-1}
  	\end{equation*}
  \end{enumerate}
\end{itemize}

\end{theorembox}
\subsection{扩展卡尔曼滤波}
扩展卡尔曼滤波做了两件事:
\begin{enumerate}
  \item 以高斯分布的方式估计$x_k$
  \item 以线性系统的方式估计非线性系统
\end{enumerate}

所以在预测过程中,首先要在当前状态和输入的附近线性化系统方程。
\begin{theorembox}
	\begin{equation*}
		F_k  = \left .\frac{\partial f}{\partial x}\right |_{\hat{x}_k, u_k}
	\end{equation*}
	使用$F_k$来更新误差方差矩阵\\
\end{theorembox}
在测量更新过程中,线性化系统输出方程。
\begin{theorembox}
	\begin{equation*}
		H_{k+1} = \left .\frac{\partial h}{\partial x}\right |_{\hat{x}_{k+1|k},u_{k+1}}
	\end{equation*}
	使用$H_{k+1}$来进行状态更新和误差方差矩阵的更新。
\end{theorembox}

\chapter{最优控制}
\section{函数最优化问题}
% 引用MathTools中的最优化方法
求解方式详见\ref{MathTools:chap:optimization}章,包括:
\begin{itemize}
    \item 原始牛顿法:参考\ref{MathTools:sec:newton_method}
    \item 雅可比矩阵和Hessian矩阵:参考\ref{MathTools:sec:jacobian_hessian}
    \item 高斯-牛顿法:参考\ref{MathTools:sec:gauss_newton}
    \item 拟牛顿法:参考\ref{MathTools:sec:quasi_newton}
\end{itemize}
\section{LQR}
\begin{theorembox}
	首先离线计算出LQR控制率:
	\begin{itemize}
		\item 从初始矩阵开始 $P_0 = Q_f$
		\item Riccati 递归:$P_{j+1} = Q + A^TP_jA - A^TP_jB(R+B^TP_jB)^{-1}B^TP_jA$
		\item 计算最优反馈增益 $K_j = (R+ B^TP_{j-1}B)^{-1}B^TP_{j-1}A$
		\item 有些教材里使用了Riccati-Lyapunov递归公式:$P_{j+1} = Q + K^TRK + (A+BK)^TP_j(A+BK)$。与Riccati公式等效。
	\end{itemize}
	应用LQR控制率:
	\begin{itemize}
		\item $u_k^\star = -K_{N-k}x_k^\star$
		\item $x_{k+1}^\star  = Ax_k^\star +Bu_k^\star$
	\end{itemize}
\end{theorembox}

\section{iLQR}
iLQR主要为了解决模型非线性,代价函数非线性的问题。通过将动力学转换为线性系统,将代价函数转换为二次系统,来求解
问题描述:有非线性离散系统
$$
\begin{aligned}
	& \min_{x_{0:N},u_{0:N-1}} l_N(x_N) + \sum^{N-1}_{k=0}l_k(x_k,u_k,\Delta t)\\
	& \text{subject to}\\
	& x_{k+1} = f(x_k,u_k,\Delta t), k = 1,\cdots,N-1 \\
	& g_k(x_k,u_k)\{<=0\},\forall k \\
	& h_k(x_k,u_k)=0,\forall k
\end{aligned}
$$

针对带约束优化问题,请阅读\ref{MathTools::sec::AL}部分的增广拉格朗日法的内容。

重要计算步骤:
\begin{enumerate}
	\item 可以通过泰勒展开在标称轨迹附近展开代价函数$l(x,u)$:
		\begin{gather*}
			l_x \equiv \left. \frac{\partial l}{\partial x}\right |_{x_k,u_k} \rightarrow Q_kx_k + q_k \\
			l_u \equiv \left. \frac{\partial l}{\partial u}\right |_{x_k,u_k} \rightarrow R_ku_k + r_k \\
			l_{xx} \equiv \left. \frac{\partial ^2 l}{\partial ^2 x}\right |_{x_k,u_k} \rightarrow Q_k \\
			l_{uu} \equiv \left. \frac{\partial ^2 l}{\partial ^2 u}\right |_{x_k,u_k} \rightarrow R_k \\
			l_{xu} \equiv \left. \frac{\partial ^2 l}{\partial x \partial u}\right |_{x_k,u_k} \rightarrow H_k \\
			l_{ux} \equiv \left. \frac{\partial ^2 l}{\partial u \partial x}\right |_{x_k,u_k} \rightarrow H^T_k
		\end{gather*}
	\item 可以通过二次型估计状态代价函数$V_k(x_k)$在标称轨迹附近的导数:
\begin{equation*}
	\delta V_k(x_k) = p_k^T\delta x_k + \frac{1}{2} \delta x_k^TP_k\delta x_k
\end{equation*}
	\item 类似的,可以通过二次型估计动作状态代价函数$Q_k(x_t,u_t)$:
	\begin{equation*}
		\delta Q_k(x_t,u_t) = \frac{1}{2}
		\begin{bmatrix}
			\delta x_k \\ \delta u_k
		\end{bmatrix}^T
		\begin{bmatrix}
			Q_{xx} & Q_{xu} \\ Q_{ux} & Q_{xx}
		\end{bmatrix}
		\begin{bmatrix}
			\delta x_k \\ \delta u_k
		\end{bmatrix} +
		\begin{bmatrix}
			Q_x \\ Q_u
		\end{bmatrix}^T
		\begin{bmatrix}
			\delta x_k \\ \delta u_k
		\end{bmatrix}
	\end{equation*}
\end{enumerate}
且满足如下关系
\begin{gather*}
	Q_x = l_x + p_{k+1}A_k + c_x^T(\lambda + I_\mu c)\\
	Q_u = l_u + p_{k+1}B_k + c_u^T(\lambda + I_\mu c)\\
	Q_{xx} = l_{xx} + A_k^TP_{k+1}A_k + c_x^TI_\mu c_x\\
	Q_{uu} = l_{uu} + B^T_kP_{k+1}B_k + c_u^TI_\mu c_u\\
	Q_{ux} = l_{ux} + B^T_kP_{k+1}A_k + c_u^TI_\mu c_x
\end{gather*}

初始化末状态的状态代价函数
\begin{gather*}
	p_N = (l_N)_x + (c_N)^T_x(\lambda + I_{\mu_N}c_N)\\
	P_N = (l_N)_{xx} + (c_N)^T_xI_{\mu_N}(c_N)_x
\end{gather*}

因为Hessian矩阵很容易病态,所以我们引入带正则化项的控制率:
\begin{equation*}
	\delta u_k^\star = -(Q_{uu} + \rho I)^{-1}(Q_{ux}\delta x_k+Q_u) \equiv K_k\delta x_k + d_k
\end{equation*}
	
\begin{theorembox}[后向传播计算控制率]
	输入:标称轨迹(X,U)
		\begin{enumerate}
			\item 在标称轨迹处线性化动力学 
			\item 在标称轨迹$(\bar{x}_k,\bar{u}_k)$处展开代价函数,计算得出$l_x, l_u, l_{xx}, l_{uu}, l_{xu}, l_{ux}$
			\item 计算得到得到$Q_x, Q_u, Q_{xx}, Q_{uu}, Q_{ux}$
			\item 得到控制率$\delta u^\star = K_k\delta x + d_k$
			\item 更新状态代价函数
			\begin{gather*}
				p_k = Q_x + K_k^TQ_{uu}d_k + K_k^TQ_u + Q^T_{ux}d_k\\
				P_k = Q_{xx} + K_k^TQ_{uu}K_k + K_k^TQ_{ux} + Q^T_{ux}K_k \\
				\Delta V_k = d_k^TQ_u + \frac{1}{2}d_k^TQ_{uu}d_k
			\end{gather*}
		\end{enumerate}
\end{theorembox}

\begin{theorembox}[前向传播rollout轨迹]
	输入:控制率,初始状态
		\begin{enumerate}
			\item $u_k = \bar{u}_k + K_k(\bar{x}_k - x_k) + \alpha d_k$
			\item $x_{k+1} = f(x_k,u_k)$
			\item 累计代价 $J += l(x_k,u_k)$
		\end{enumerate}
	收敛检查
\end{theorembox}
% todo : 补全iLQR文档

\subsection{Square-Root后向传播}
该方法主要使用Cholesky分解或QR分解来更具鲁棒性的求解递归问题。
\section{DDP}

\chapter{轨迹优化}
\section{概述}
\section{直接法}
\subsection{Single Shooting}
\subsection{Multiple Shooting}
\subsection{Collocation Method}
\section{间接法}





