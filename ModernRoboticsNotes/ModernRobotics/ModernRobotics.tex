\chapter{机器人运动学基础}
\chapter{逆向运动学解法}
\section{逆运动学的基础知识}
机器人的逆向运动学是,已知末端的位置和姿态,以及所有连杆的几何参数下,求解关节的位置。逆运动学求解通常有两大类方法:解析法、数值法。
\section{解析法}
特点:运算速度快(达到us级),通用性差,可以分为代数法与几何法进行求解。

串联机械臂有逆运动学解析解的充分条件是满足Pieper准则。即如果机器人满足两个充分条件中的一个,就会得到封闭解,这两个条件是:
\begin{itemize}
  \item 三个相邻关节轴相交于一点;
  \item 三个相邻关节轴相互平行。
\end{itemize}

现在的大多数商品化的工业机器人在设计构型时,都会尽可能满足满足Pieper准则,因为解析法求解能够很快的使用较少的算力,使用较低成本的控制器就能求解,之后随着芯片算力的提升,感觉在未来,机器人公司也会在是否采用满足解析解的构型和采用特定构型并开发对应的逆解算法之间找一个平衡。

以PUMA560机器人为例,它的最后3个关节轴相交于一点。我们运用Pieper方法解出它的封闭解。对于UR5机械臂,其第2、第3、第4关节轴平行,满足Pieper准则其中的一条,即三个相邻的关节轴两两平行。

\section{数值法(Numerical Solution)}
特点:通用性高,但是求解速度较慢(ms级)。

除了一些特殊的机械臂构型外,机械臂逆运动学问题很难用解析解求解,因此在许多情况下会使用数值解求解。

通常设定一个优化目标函数,是把逆解求解问题转化为一个优化问题求数值解。

Newton-Raphson(NR)是数值解的一种方法。它需要基本的雅可比矩阵。然而,当且仅当原始方程的函数具有逆函数,且原始方程可解时,NR方法才会成功。从运动学的角度来看,前一个条件意味着机器人需要非冗余,机器人在从初始配置到最终配置的运动过程中不通过奇异点。后一个条件意味着机械臂的期望位置和方向需要在机器人的工作空间内,是可解的。由于这些限制,NR方法不能保证全局收敛性,因此它在很大程度上取决于初始值。

奇异性问题与基本雅可比矩阵的性质密切相关,这主要在微分逆运动学(differential inverse kinematics)相关领域 中进行了讨论,Whitney提出了使用Moore–Penrose的广义逆矩阵来解决这个问题。无论原始雅可比矩阵的秩如何,都通过构型的最小偏差使约束方程的残差最小化。Nakamura and Hanafusa指出,Whitney的方法不能解决构型在奇异点附近抖动的问题,并提出了引入阻尼因子的奇异鲁棒逆矩阵(singularity-robust inverse matrix)。Wampler 也提出了一种类似的方法,并提到它涉及到 Levenberg–Marquardt(LM)方法的框架。

可解性的问题和奇异点的问题一样需要考虑。大多数情况下,很难提前知道方程是否是可解的。一个合理的思想是用残差极小化代替逆运动学中的根查找问题。基于这一想法,有使用了最陡下降( steepest descent,SD)和变量度量(Variable metric,VM)的方法。但前者的收敛速度较慢,而后者的可靠性较低,因此经常处于局部极小值。合理快速的解决方案是一类在每一步迭代中利用DIK的梯度方法。LM方法在其中具有较高的计算稳定性。虽然LM方法的收敛性能取决于阻尼因子的选择,但这个问题到目前为止还没有得到充分的讨论。Tomomichi Sugihara提出了一种选择LM方法的阻尼因子的方法,该方法对奇异性、可解性和快速收敛问题具有鲁棒性。通过一种相当简单的利用残差平方范数作为阻尼因子的方法来实现鲁棒性和收敛性,并通过对阻尼因子略有偏置来解决在奇异点附近,计算不稳定的问题。
\subsection{雅可比求逆法Jacobian Inverse}
根据微分运动方程,可得
\begin{equation*}
  \dot{x} = J \dot{q}
\end{equation*}
对雅可比矩阵求逆可得
\begin{equation*}
  \dot{q} = J^{-1}\dot{x}
\end{equation*}
基于微分运动学的思想,机械臂末端点的世界坐标系中的运动可以近似用关节的运动叠加得到,$\Delta x$ 越小,线性关系越准确,即假设机器人从初始位置 $q_s$ 一步步地运动到目标位置 $q_e$ ,每一步的步长为 $\Delta \theta$ 。
\begin{equation*}
  \Delta \theta = \alpha J^{-1}\Delta x
\end{equation*}
\subsubsection{冗余机械臂的情况}
针对冗余机械臂的情况,即雅可比矩阵为非方阵或者不是满秩矩阵,对雅可比矩阵求的是伪逆,Whitney提出了使用Moore–Penrose即$J^{\dagger}$表示,其中$J^{\dagger} = J^T(JJ^T)^{-1}$。此时逆解目标是求解使关节速度的二次型泛函最小的解,最终得到下面的式子计算伪逆。
\begin{theorembox}
\begin{equation*}
  \Delta \theta = \alpha J^{\dagger}\Delta x
\end{equation*}
\end{theorembox}
针对伪逆,我们可以进一步扩展,得到如下条件
\begin{theorembox}
\begin{equation*}
  \forall \dot{q}, J(I-J^{\dagger}J)\dot{q}_0 = 0
\end{equation*}
\end{theorembox}
\subsubsection{梯度投影法}
梯度投影法将附加优化目标的梯度投影到雅可比矩阵的零空间中,利用零空间的性质,从而在保证主任务(跟踪机械臂末端位姿)的前提下,实现对附加目标的优化。
\begin{theorembox}
	新的关节空间解:
	\begin{equation*}
  \dot{q} = J^{\dagger}\dot{x} +(I_n - J^{\dagger}J)\dot{q}_0
\end{equation*}
其中矩阵$(I_n - J^{\dagger}J)$是J的零空间上进行投影的投影矩阵。针对上面的子目标$\dot{q}_0$
\begin{equation*}
  \dot{q}_0 = k_0\left(\frac{\partial \omega(q)}{\partial q}\right)^T
\end{equation*}
其中$k_0 > 0$,且$\omega(q)$是关节变量的次级目标函数,可以根据不同的任务,确定不同的目标函数
\begin{itemize}
	\item 可操作度 $\omega (q) = \sqrt{det(J(q)J^T(q))}$
	\item 与关节极限的距离 $\omega (q) = -\frac{1}{2n}\sum^n_{i=1}\left(\frac{q_i - \bar{q_i}}{q_{iM} - q_{im}}\right)^2$
	\item 与障碍物之间的距离 $\omega (q) = \min_{p,o}\Vert p(q) - o \Vert$
\end{itemize}
\end{theorembox}
\subsection{雅可比矩阵求逆法存在的问题}
\begin{enumerate}
  \item 从原理上讲,使用这个方法最明显的一点要求是$dx$不能过大。因为Jacobian是随着关节位置变化不断在变化的,一旦关节位置变化很大,算出来的Jacobian Inverse就不再准确了。通常我们可以用轨迹线性插值(linear interpolation)或限制$dx$的大小(clamping)来避免。
  \item 雅可比矩阵求逆运算,矩阵求逆是一个非常消耗计算资源的运算。当然,我们总是可以使用各种各样的解线性方程的方法来避开求逆运算,比如LU分解、Chelosky分解、QR分解、SVD(Singular Value Decomposition)等等。
  \item 雅可比矩阵求逆法最大的问题还是在于它无法很好地对付机器人Singularity或接近Singularity的情况。从线性方程的角度看,当机器人接近Singularity时,雅可比矩阵也越来越“病态”(ill-conditioned),很小的dx可能求得很大的dq,方程对数值误差也更加敏感;而当机器人处于Singularity时,线性方程可能无解、也可能有无数多个解。
\end{enumerate}
\section{雅可比矩阵转置法(Jacobian Transpose)}
可以用雅可比矩阵的转置可避免由雅可比伪逆矩阵所导致的机械臂奇异构型,同时,可通过确定雅可比矩阵是否具有零行来验证是否存在奇异性问题。但是,相比于雅可比伪逆法,雅可比转置法需要更多次迭代才能收敛。
\begin{theorembox}
	\begin{equation*}
  \Delta \theta = \alpha J^T\Delta x
\end{equation*}
\end{theorembox}
\subsubsection{雅可比矩阵转置法存在的问题}
它的迭代收敛速率慢些。另外,用这个方法控制的机械臂,离end effector较远的关节常常需要输出更大的扭矩。
\subsection{阻尼最小二乘法(Damped least squares)}
比较典型的是阻尼最小二乘法(Damped least squares)其最早由 C. W. Wampler 和 Y. Nakamura 等人提出,由于阻尼最小二乘法所计算出的关节角度增量稳定,因此避免了伪逆方法所产生的奇异问题,同时 S. Buss 和 J. S. Kim 在文献中指出阻尼最小二乘法总体计算性能优于雅可比伪逆法和雅可比转置法。然而,较大的阻尼常数虽然会使机械臂逆运动学解在奇异点附近表现良好,但同时会降低算法收敛速度和目标跟踪精度,并使机械臂在运动过程中产生振荡和抖动现象。
\begin{theorembox}
	\begin{equation*}
  \Delta \theta = J^T(JJ^T+ \lambda ^2I)^{-1}e
\end{equation*}
\end{theorembox}
伪逆DLS方法是DLS方法的扩展,在DLS方法下使用SVD。因此,它可以表示为
\begin{theorembox}
	\begin{equation*}
  J^T(JJ^T+ \lambda ^2I)^{-1} \longrightarrow \sum^{r}_{i=1}\frac{\sigma_i}{\sigma _i ^2+\lambda ^2}v_iu_i^T
\end{equation*}

\end{theorembox}

伪逆DLS方法在奇异点附近的表现与简单伪逆方法相似,但在奇异点附近的表现较为平滑。

\subsection{选择性阻尼最小二乘(Selectively damped least squares)}
选择性SDLS方法是伪逆DLS方法的扩展。SDLS根据达到目标位置的难度,对雅可比矩阵奇异值分解的每个奇异向量分别调整阻尼因子。SDLS的阻尼常数不仅与多体关节的当前构型有关,还与末端受动器与目标位置的相对位置有关。SDLS需要更少的迭代来收敛,不需要特别的阻尼常数,并返回在缺乏波动方面的最佳结果。该方法的性能优于其他任何一种反雅可比矩阵方法,但其缺点是由于奇异值分解的计算,其计算成本较高(在所有雅可比矩阵方法中性能时间最慢)。

相关论文有很多,这里推荐一篇TRO上的Short Paper:

Solvability-Unconcerned Inverse Kinematics by the Levenberg–Marquardt Method[J]. by Tomomichi Sugihara, 2011


\section{冗余机械臂的逆运动学解法}
基于雅可比矩阵的速度级优化则适合多种构型的机械臂,针对冗余的机械臂,主要演化出了梯度投影法、构型控制法和加权最小范数法三种优化方法。
\begin{enumerate}
  \item 梯度投影法将附加优化目标的梯度投影到雅可比矩阵的零空间中,利用零空间的性质,从而在保证主任务(跟踪机械臂末端位姿)的前提下,实现对附加目标的优化。
  \item 构型控制法是将附加优化目标与主任务合并成扩展雅可比矩阵,使其成为方阵,具有可逆矩阵。再通过直接求逆,从而获得考虑附加优化目标的关节角速度结果。
  \item 加权最小范数法则是将式$\dot{x} = J\dot{q}$看成一个带约束的优化问题,利用拉格朗日乘子法,获得公式的显示解,从而完成对特定目标的优化。
\end{enumerate}



\chapter{动力学基础}
\chapter{基于旋量理论的动力学}
\section{符号约定}
\section{}
\chapter{基于约束的动力学方程求解}
