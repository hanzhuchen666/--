\chapter{机器人运动学基础}
\chapter{逆向运动学解法}
一、逆向运动学基础知识
机器人的逆向运动学是,已知末端的位置和姿态,以及所有连杆的几何参数下,求解关节的位置。



二、两大类求解逆运动学的方法
逆运动学求解通常有两大类方法:解析法、数值法。

参考文章,机械臂的逆运动学求解有统一的方法吗,还是每个机械臂的求解方式不一样?
1.解析法(Analytical Solution)
特点:运算速度快(达到us级),通用性差,可以分为代数法与几何法进行求解。

串联机械臂有逆运动学解析解的充分条件是满足Pieper准则。即如果机器人满足两个充分条件中的一个,就会得到封闭解,这两个条件是:

三个相邻关节轴相交于一点;
三个相邻关节轴相互平行。
现在的大多数商品化的工业机器人在设计构型时,都会尽可能满足满足Pieper准则,因为解析法求解能够很快的使用较少的算力,使用较低成本的控制器就能求解,之后随着芯片算力的提升,感觉在未来,机器人公司也会在是否采用满足解析解的构型和采用特定构型并开发对应的逆解算法之间找一个平衡。

以PUMA560机器人为例,它的最后3个关节轴相交于一点。我们运用Pieper方法解出它的封闭解。对于UR5机械臂,其第2、第3、第4关节轴平行,满足Pieper准则其中的一条,即三个相邻的关节轴两两平行。



2.数值法(Numerical Solution)
特点:通用性高,但是求解速度较慢(ms级)。

除了一些特殊的机械臂构型外,机械臂逆运动学问题很难用解析解求解,因此在许多情况下会使用数值解求解。

通常设定一个优化目标函数,是把逆解求解问题转化为一个优化问题求数值解。

Newton-Raphson(NR)是数值解的一种方法。它需要基本的雅可比矩阵。然而,当且仅当原始方程的函数具有逆函数,且原始方程可解时,NR方法才会成功。从运动学的角度来看,前一个条件意味着机器人需要非冗余,机器人在从初始配置到最终配置的运动过程中不通过奇异点。后一个条件意味着机械臂的期望位置和方向需要在机器人的工作空间内,是可解的。由于这些限制,NR方法不能保证全局收敛性,因此它在很大程度上取决于初始值。

奇异性问题与基本雅可比矩阵的性质密切相关,这主要在微分逆运动学(differential inverse kinematics)相关领域 中进行了讨论,Whitney提出了使用Moore–Penrose的广义逆矩阵来解决这个问题。无论原始雅可比矩阵的秩如何,都通过构型的最小偏差使约束方程的残差最小化。Nakamura and Hanafusa指出,Whitney的方法不能解决构型在奇异点附近抖动的问题,并提出了引入阻尼因子的奇异鲁棒逆矩阵(singularity-robust inverse matrix)。Wampler 也提出了一种类似的方法,并提到它涉及到 Levenberg–Marquardt(LM)方法的框架。

可解性的问题和奇异点的问题一样需要考虑。大多数情况下,很难提前知道方程是否是可解的。一个合理的思想是用残差极小化代替逆运动学中的根查找问题。基于这一想法,有使用了最陡下降( steepest descent,SD)和变量度量(Variable metric,VM)的方法。但前者的收敛速度较慢,而后者的可靠性较低,因此经常处于局部极小值。合理快速的解决方案是一类在每一步迭代中利用DIK的梯度方法。LM方法在其中具有较高的计算稳定性。虽然LM方法的收敛性能取决于阻尼因子的选择,但这个问题到目前为止还没有得到充分的讨论。Tomomichi Sugihara提出了一种选择LM方法的阻尼因子的方法,该方法对奇异性、可解性和快速收敛问题具有鲁棒性。通过一种相当简单的利用残差平方范数作为阻尼因子的方法来实现鲁棒性和收敛性,并通过对阻尼因子略有偏置来解决在奇异点附近,计算不稳定的问题。



2.1 雅可比矩阵求逆法( Jacobian Inverse)
根据微分运动方程,可得

对雅可比矩阵求逆可得


如果机械臂的初始关节状态 
 已知,最终的目标关节位置可以通过速度对时间的积分进行计算。

积分的计算可以通过数值方法对时间离散来实现,最简单是基于欧拉积分法,给定一个积分间隔 
 ,如果 
 时刻的关节位置与速度已知, 
 时刻的关节位置可以通过以下实现。

因此我们的逆解求解的关注点便聚焦于求解 
,即 
 。



基于微分运动学的思想,机械臂末端点的世界坐标系中的运动 
 可以近似用关节的运动 
 叠加得到,
 越小,线性关系越准确,即假设机器人从初始位置 
 一步步地运动到目标位置 
,每一步的步长为 
 。

一般迭代求解会对求解的步长做处理,即增加一个变量
,这样可以加快求解速度。最后得到如下的迭代公式


冗余机械臂的情况:

针对冗余机械臂的情况,即雅可比矩阵为非方阵或者不是满秩矩阵,对雅可比矩阵求的是伪逆,Whitney提出了使用Moore–Penrose即 
 表示,其中
。

逆解目标是求解使关节速度的二次型泛函最小的解

二次型: 
 上的一个二次型是一个定义在 
 上的函数,表达式为 
 ,其中 
 是一个 
 对称矩阵(一定要注意 
 是对称矩阵,不是一个随便的矩阵),矩阵 
 称为关于二次型的矩阵(二次型矩阵)
最简单的一个二次型是 

注意矩阵 
 中对角线上的元素涉及到平方项,斜对角线上的元素涉及到交叉乘积项


用拉格朗日乘数法(Lagrange Multiplier),这个问题可以变成:

要求极值,必先求导。拉格朗日乘数法是分别对 
 和 
 求导,导数为 0 时可求得极值


解上面两个方程,即可求出


不难验证 
 。用这个 
 求解出的 
 即为满足条件的最小关节运动速度。

最终得到下面的式子计算伪逆。


针对伪逆,我们可以进一步扩展,得到如下条件


便因此提出了梯度投影法,详见文章裕如:冗余机械臂求解逆运动学解——梯度投影法

梯度投影法将附加优化目标的梯度投影到雅可比矩阵的零空间中,利用零空间的性质,从而在保证主任务(跟踪机械臂末端位姿)的前提下,实现对附加目标的优化。



实际的求解过程:
先通过计算机械臂运动学雅克比矩阵,将关节速度映射到任务空间速度。然后,通过计算雅克比的逆矩阵(Jacobian inverse)将机械臂末端执行器位姿误差(即任务误差)映射到关节位置,进而对关节位置矢量进行迭代更新以减小任务误差。


算法的流程为:关节控制模块(Joint control)把雅各比求逆计算出来的 
 发送到每一个关节的执行控制器中(比如舵机、关节电机等);这些执行控制器最终把各个关节控制到位置 
 (好的控制器,大部分情况 
 和 
 应当非常接近)。

同时,在反馈回路中,我们又从这几个关节中利用传感器(多为位置的编码器)读取到实际位置信息 
 ,再用正运动学求出了此时的实际末端执行器的位置 
 ,反馈到前面进行求误差运算求得 
 。

其中 
 , 
 , 

通常设置一个 
 ,控制器需要运行迭代几次这个控制回路才能将 
 降到接近0(让 
 逼近 
 ),因此这个方法也被称为迭代法。





雅可比矩阵求逆法存在的问题

1、

从原理上讲,使用这个方法最明显的一点要求是—— 
 不能过大。因为Jacobian是随着关节位置变化不断在变化的,一旦关节位置变化很大,算出来的Jacobian Inverse就不再准确了。

通常我们可以用轨迹线性插值(linear interpolation)或限制 
 的大小(clamping)来避免。

2、

雅可比矩阵求逆运算,矩阵求逆是一个非常消耗计算资源的运算。

当然,我们总是可以使用各种各样的解线性方程的方法来避开求逆运算,比如LU分解、Chelosky分解、QR分解、SVD(Singular Value Decomposition)等等。

奇点的存在使雅可比矩阵求逆过程相当复杂。提出了奇异值分解(SVD)作为利用伪逆矩阵的雅可比矩阵方法的另一种变体。SVD为矩阵的基本子空间提供了标准正交基。

形式上,
的雅可比矩阵的奇异值分解
是式子
的因式分解,
是一个
酉正交矩阵,
是一个
的对角线上有非负实数的矩形对角矩阵,
(
的转置)是一个
酉正交矩阵。D矩阵
的对角项称为
的奇异值。

雅可比伪逆可以用SVD表示为
。其中,伪逆
由
表示。

雅可比矩阵求逆法最大的问题还是在于它无法很好地对付机器人Singularity或接近Singularity的情况。从线性方程的角度看,当机器人接近Singularity时,雅可比矩阵也越来越“病态”(ill-conditioned),很小的dx可能求得很大的dq,方程对数值误差也更加敏感;而当机器人处于Singularity时,线性方程可能无解、也可能有无数多个解。

2.2 雅可比矩阵转置法(Jacobian Transpose)
因此,可以用雅可比矩阵的转置可避免由雅可比伪逆矩阵所导致的机械臂奇异构型,同时,可通过确定雅可比矩阵是否具有零行来验证是否存在奇异性问题。但是,相比于雅可比伪逆法,雅可比转置法需要更多次迭代才能收敛。

首先确定求解的目标:

用正运动学的表达式可以把 
 用 
 表示

由于正运动学的表达式 
 通常比较复杂,导数极值法在这里不太好用。所以我们使用梯度下降法(Gradient Descent),求上面表达式的梯度

最终得到式子


其中 
 是梯度下降法的一个系数,叫“步长”(step size)或“学习速率”(learning rate)。这个值太小,则迭代速度可能太慢;这个值太大,则可能“走过头”或要多绕一些弯路。





雅各比矩阵的转置也可以用来表示关节力与末端执行器力之间的关系,推导如下,根据虚功原理,

两
边
转
置
两
边
转
置

从受力角度分析,雅克比转置的方法是给机械臂每个关节一个朝目标点移动的力矩,这样就相当于一直拉着机械臂末端往目标点拖,机械臂最终将收敛到目标位置。

参考文献,Introduction to Inverse Kinematics with Jacobian Transpose, Pseudoinverse and Damped Least Squares methods[J]. by Samuel R. Buss.


雅可比矩阵转置法存在的问题

它的迭代收敛速率反而慢些。另外,用这个方法控制的机械臂,离end effector较远的关节常常需要输出更大的扭矩。



2.3 阻尼最小二乘法(Damped least squares)


比较典型的是阻尼最小二乘法(Damped least squares)其最早由 C. W. Wampler 和 Y. Nakamura 等人提出,由于阻尼最小二乘法所计算出的关节角度增量稳定,因此避免了伪逆方法所产生的奇异问题,同时 S. Buss 和 J. S. Kim 在文献中指出阻尼最小二乘法总体计算性能优于雅可比伪逆法和雅可比转置法。然而,较大的阻尼常数虽然会使机械臂逆运动学解在奇异点附近表现良好,但同时会降低算法收敛速度和目标跟踪精度,并使机械臂在运动过程中产生振荡和抖动现象。



DLS解可以表示为


其中
是一个非零阻尼常数。为了使上述式子数值稳定,阻尼常数必须谨慎选择。

伪逆DLS方法是DLS方法的扩展,在DLS方法下使用SVD。因此,它可以表示为

伪逆DLS方法在奇异点附近的表现与简单伪逆方法相似,但在奇异点附近的表现较为平滑。

更具体地说,这两种方法的雅可比矩阵都由表达式 
 求逆。但在伪逆DLS情况下, 
 ,而在简单的伪逆方法中, 
 ,当 
 趋近于0时, 
 不稳定。

选择性阻尼最小二乘(Selectively damped least squares)

选择性SDLS方法是伪逆DLS方法的扩展。SDLS根据达到目标位置的难度,对雅可比矩阵奇异值分解的每个奇异向量分别调整阻尼因子。SDLS的阻尼常数不仅与多体关节的当前构型有关,还与末端受动器与目标位置的相对位置有关。SDLS需要更少的迭代来收敛,不需要特别的阻尼常数,并返回在缺乏波动方面的最佳结果。该方法的性能优于其他任何一种反雅可比矩阵方法,但其缺点是由于奇异值分解的计算,其计算成本较高(在所有雅可比矩阵方法中性能时间最慢)。






相关论文有很多,这里推荐一篇TRO上的Short Paper:

Solvability-Unconcerned Inverse Kinematics by the Levenberg–Marquardt Method[J]. by Tomomichi Sugihara, 2011


三、逆向运动学的实际应用
摘自邱博的回答 , 机器人厂商的逆解是怎么做的?
成熟机器人厂商的逆解都是自己写的解析解。对于绝大多数六轴及以下自由度机器人,运动学正逆解非常简单,可能就几百行代码。

数值解其实就是用数值优化的方式迭代求解,建模方式(坐标系选取)、迭代参数(步长、终止条件等)、优化方法(简单的牛顿欧拉,还是其他)都会极大影响收敛效率。

这就是为什么同为数值解,tracik 比 kdl 效果好很多(tracik 甚至直接使用 kdl 的数据结构); ikfast 是一个自动生成不同机器人的运动学解析解代码的工具,其使用 python 符号计算工具,通过遍历搜索的方式寻找可求解的项。 ikfast 的最大贡献是把很多前人论文里的自动求解算法给写成了程序,并开源出来了。但是,如果你真的去看 ikfast 生成的代码,可能一个六轴机械臂的运动学正逆解就有上万行。通用往往意味着不是最优。如果自己生产机器人、设计机器人控制器,那么自己写运动学会是更优的选择。


在具体的实现方法上,

3.1 冗余机械臂的逆运动学解法
基于雅可比矩阵的速度级优化则适合多种构型的机械臂,针对冗余的机械臂,主要演化出了梯度投影法、构型控制法和加权最小范数法三种优化方法。

3.1.1 梯度投影法
梯度投影法将附加优化目标的梯度投影到雅可比矩阵的零空间中,利用零空间的性质,从而在保证主任务(跟踪机械臂末端位姿)的前提下,实现对附加目标的优化。

3.1.2 构型控制法
构型控制法是将附加优化目标与主任务合并成扩展雅可比矩阵,使其成为方阵,具有可逆矩阵。再通过直接求逆,从而获得考虑附加优化目标的关节角速度结果。

3.1.3 加权最小范数法
加权最小范数法则是将式 
 看成一个带约束的优化问题,利用拉格朗日乘子法,获得公式的显示解,从而完成对特定目标的优化。

以上三种方法从不同角度对上式进行求解,方法简单直观,但都存在奇异性和累积误差等问题。





一些还未解决的问题:

在解析解求解时存在多解的问题,如何选择一组解
数值解存在对初值敏感的问题,如何解决


2022.9.11更新

在浏览文章时,刷到了一篇讲游戏里面物体求解逆运动学文章:爱吃菠萝不吃萝卜:【游戏开发】逆向运动学(IK)详解,里面的参考文献为一篇逆运学综述,没想到游戏领域这方面也有很多应用:Inverse Kinematics Techniques in Computer Graphics: A Survey
\chapter{动力学基础}
\chapter{基于旋量理论的动力学}
\section{符号约定}
\section{}
\chapter{基于约束的动力学方程求解}
